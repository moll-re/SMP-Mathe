\chapter{Vektorielle Geometrie}


\section{Vektoren}
\begin{Bws}
Ein Vektor ist Element eines Vektorraums.
\end{Bws}\\
Vektorräume, wir erinnern uns zurück. Verknüpfungen, inverse Elemente und die dazugehörenden Gesetze, konsequente Definitionen und mathematische Korrektheit, die guten alten Zeiten...\\
Tatsächlich kann ein Vektor in den meisten Fällen als Verschiebung bezeichnet werden, \textbf{nicht aber als Pfeil oder Strich!}\\
\subsection{Besondere Vektoren}
\subsubsection{Der Ortsvektor}
Der Vektor von $O$ auf den Punkt $P$, geschrieben als $\vec{OP}$ oder $\vec{o}$.\\
Hat $P$ die Koordinaten $(P_1|P_2|P_3)$, so besitzt $\vec{o}$ die Darstellung $\left(\begin{array}{c} P_1 \\ P_1 \\ P_3\end{array}\right)$.
\subsubsection{Der Nullvektor}
Der Vektor mit Wert $\left(\begin{array}{c} 0 \\ 0 \\ 0\end{array}\right)$, er hat keine und alle Richtungen zugleich.
