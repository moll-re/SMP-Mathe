\chapter{ANHANG: Physik}


\section{La physique des particules}

Le modele standard de la physique dans sa beauté incontestée.\\\\

\usetikzlibrary{calc,positioning,shadows.blur,decorations.pathreplacing}

\tikzset{%
        brace/.style = { decorate, decoration={brace, amplitude=5pt} },
       mbrace/.style = { decorate, decoration={brace, amplitude=5pt, mirror} },
        label/.style = { black, midway, scale=0.5, align=center },
     toplabel/.style = { label, above=.5em, anchor=south },
    leftlabel/.style = { label,rotate=-90,left=.5em,anchor=north },
  bottomlabel/.style = { label, below=.5em, anchor=north },
        force/.style = { rotate=-90,scale=0.4 },
        round/.style = { rounded corners=2mm },
       legend/.style = { right,scale=0.4 },
        nosep/.style = { inner sep=0pt },
   generation/.style = { anchor=base }
}

\newcommand\particle[7][white]{%
  \begin{tikzpicture}[x=1cm, y=1cm]
    \path[fill=#1,blur shadow={shadow blur steps=5}] (0.1,0) -- (0.9,0)
        arc (90:0:1mm) -- (1.0,-0.9) arc (0:-90:1mm) -- (0.1,-1.0)
        arc (-90:-180:1mm) -- (0,-0.1) arc(180:90:1mm) -- cycle;
    \ifstrempty{#7}{}{\path[fill=purple!50!white]
        (0.6,0) --(0.7,0) -- (1.0,-0.3) -- (1.0,-0.4);}
    \ifstrempty{#6}{}{\path[fill=green!50!black!50] (0.7,0) -- (0.9,0)
        arc (90:0:1mm) -- (1.0,-0.3);}
    \ifstrempty{#5}{}{\path[fill=orange!50!white] (1.0,-0.7) -- (1.0,-0.9)
        arc (0:-90:1mm) -- (0.7,-1.0);}
    \draw[\ifstrempty{#2}{dashed}{black}] (0.1,0) -- (0.9,0)
        arc (90:0:1mm) -- (1.0,-0.9) arc (0:-90:1mm) -- (0.1,-1.0)
        arc (-90:-180:1mm) -- (0,-0.1) arc(180:90:1mm) -- cycle;
    \ifstrempty{#7}{}{\node at(0.825,-0.175) [rotate=-45,scale=0.2] {#7};}
    \ifstrempty{#6}{}{\node at(0.9,-0.1)  [nosep,scale=0.17] {#6};}
    \ifstrempty{#5}{}{\node at(0.9,-0.9)  [nosep,scale=0.2] {#5};}
    \ifstrempty{#4}{}{\node at(0.1,-0.1)  [nosep,anchor=west,scale=0.25]{#4};}
    \ifstrempty{#3}{}{\node at(0.1,-0.85) [nosep,anchor=west,scale=0.3] {#3};}
    \ifstrempty{#2}{}{\node at(0.1,-0.5)  [nosep,anchor=west,scale=1.5] {#2};}
  \end{tikzpicture}
}

\begin{document}
\begin{tikzpicture}[x=1.2cm, y=1.2cm]
  \draw[round] (-0.5,0.5) rectangle (4.4,-1.5);
  \draw[round] (-0.6,0.6) rectangle (5.0,-2.5);
  \draw[round] (-0.7,0.7) rectangle (5.6,-3.5);

  \node at(0, 0)   {\particle[gray!20!white]
                   {$u$}        {up}       {$2.3$ MeV}{1/2}{$2/3$}{R/G/B}};
  \node at(0,-1)   {\particle[gray!20!white]
                   {$d$}        {down}    {$4.8$ MeV}{1/2}{$-1/3$}{R/G/B}};
  \node at(0,-2)   {\particle[gray!20!white]
                   {$e$}        {éléctron}       {$511$ keV}{1/2}{$-1$}{}};
  \node at(0,-3)   {\particle[gray!20!white]
                   {$\nu_e$}    {$e$ neutrino}         {$<2$ eV}{1/2}{}{}};
  \node at(1, 0)   {\particle
                   {$c$}        {charm}   {$1.28$ GeV}{1/2}{$2/3$}{R/G/B}};
  \node at(1,-1)   {\particle
                   {$s$}        {strange}  {$95$ MeV}{1/2}{$-1/3$}{R/G/B}};
  \node at(1,-2)   {\particle
                   {$\mu$}      {muon}         {$105.7$ MeV}{1/2}{$-1$}{}};
  \node at(1,-3)   {\particle
                   {$\nu_\mu$}  {$\mu$ neutrino}    {$<190$ keV}{1/2}{}{}};
  \node at(2, 0)   {\particle
                   {$t$}        {top}    {$173.2$ GeV}{1/2}{$2/3$}{R/G/B}};
  \node at(2,-1)   {\particle
                   {$b$}        {bottom}  {$4.7$ GeV}{1/2}{$-1/3$}{R/G/B}};
  \node at(2,-2)   {\particle
                   {$\tau$}     {tau}          {$1.777$ GeV}{1/2}{$-1$}{}};
  \node at(2,-3)   {\particle
                   {$\nu_\tau$} {$\tau$ neutrino}  {$<18.2$ MeV}{1/2}{}{}};
  \node at(3,-3)   {\particle[orange!20!white]
                   {$W^{\hspace{-.3ex}\scalebox{.5}{$\pm$}}$}
                                {}              {$80.4$ GeV}{1}{$\pm1$}{}};
  \node at(4,-3)   {\particle[orange!20!white]
                   {$Z$}        {}                    {$91.2$ GeV}{1}{}{}};
  \node at(3.5,-2) {\particle[green!50!black!20]
                   {$\gamma$}   {photon}                        {}{1}{}{}};
  \node at(3.5,-1) {\particle[purple!20!white]
                   {$g$}        {gluon}                    {}{1}{}{color}};
  \node at(5,0)    {\particle[gray!50!white]
                   {$H$}        {Higgs}              {$125.1$ GeV}{0}{}{}};
  \node at(6.1,-3) {\particle
                   {}           {graviton}                       {}{}{}{}};

  \node at(4.25,-0.5) [force]      {interaction nucléaire forte (couleur)};
  \node at(4.85,-1.5) [force]    {interaction éléctromagnétique (charge)};
  \node at(5.45,-2.4) [force] {interaction nucléaire faible (isospin faible)};
  \node at(6.75,-2.5) [force]        {interaction gravitationelle (masse)};

  \draw [<-] (2.5,0.3)   -- (2.7,0.3)          node [legend] {charge};
  \draw [<-] (2.5,0.15)  -- (2.7,0.15)         node [legend] {couleurs};
  \draw [<-] (2.05,0.25) -- (2.3,0) -- (2.7,0) node [legend]   {masse};
  \draw [<-] (2.5,-0.3)  -- (2.7,-0.3)         node [legend]   {spin};

  \draw [mbrace] (-0.8,0.5)  -- (-0.8,-1.5)
                 node[leftlabel] {6 quarks\\(+6 anti-quarks)};
  \draw [mbrace] (-0.8,-1.5) -- (-0.8,-3.5)
                 node[leftlabel] {6 leptons\\(+6 anti-leptons)};
  \draw [mbrace] (-0.5,-3.6) -- (2.5,-3.6)
                 node[bottomlabel]
                 {12 fermions\\(+12 anti-fermions)\\ masse++ $\to$};
  \draw [mbrace] (2.5,-3.6) -- (5.5,-3.6)
                 node[bottomlabel] {5 bosons\\(+1 charge opposée $W$)};

  \draw [brace] (-0.5,.8) -- (0.5,.8) node[toplabel]         {matiere ordinaire};
  \draw [brace] (0.5,.8)  -- (2.5,.8) node[toplabel]         {matiere instable};
  \draw [brace] (2.5,.8)  -- (4.5,.8) node[toplabel]          {transmetteurs de force};
  \draw [brace] (4.5,.8)  -- (5.5,.8) node[toplabel]       {bosons\\Goldstone};
  \draw [brace] (5.5,.8)  -- (7,.8)   node[toplabel] {en dehors\\du mod�le standard};

  \node at (0,1.2)   [generation] {1\tiny st};
  \node at (1,1.2)   [generation] {2\tiny nd};
  \node at (2,1.2)   [generation] {3\tiny rd};
  \node at (2.8,1.2) [generation] {\tiny generation};
\end{tikzpicture}

\section{Interaction gravitationelle}

\begin{Definition}
L'interaction gravitationnelle est une force toujours \textcolor{red}{attractive} qui agit sur tout ce qui poss�de une masse, mais avec une intensité extr�mement faible (c'est l'interaction la plus faible). Son domaine d'action est \textcolor{red}{l'infini}.\\
Un corps est considéré ponctuel si sa taille $\leq \dfrac{\text{distance d'observation}}{100}$
\\\\
$$ \vv{F_{g}} =  -\dfrac{G \cdot m_{a} \cdot m_{b}}{r^2} \cdot \vv{u_{AB}}$$  \\

Si: \qquad \qquad $r=$AB \qquad \qquad $G=6,67\cdot10^{-11}$(S.I) \qquad \qquad $\vv{u_{AB}} \to$ vecteur normé
\end{Definition}

\subsection{Le champ de gravitation}

Tout objet de Masse M et d'origine spaciale $O$ crée autour de lui un champ gravitationnel.\\
En un point quelconque $P$, ce champ s'écrit $\vv{\mathcal{G}}_{(P)}$. \\
Un deuxi�me objet de masse $m$ placé en ce point P est soumis a la force de gravitation:
$$\vv{F}_{O/P} = m \cdot \vv{\mathcal{G}}_{(P)} $$
D'ou on peut tirer la formule pour le champ de gravitation d'un objet considéré ponctuel de masse $M$ a une distance $d$:
$${\mathcal{G}}_{o} = \dfrac{G\cdot M}{d^2} $$


\section{Interaction électromagnétique}

\begin{Definition}
L'interaction éléctromagn�tique est une force \textcolor{red}{attractive} ou \textcolor{red}{répulsive} qui agit sur tout ce qui poss�de une charge éléctrique. Son domaine d'action est également \textcolor{red}{l'infini.}
\end{Definition}

\subsection{Le champ électrique}

\begin{Definition}
La loi de Coulomb\\
Dans le vide, 2 corps ponctuels A et B de charges $q_{a}$ et $q_{b}$ exercent l'un sur l'autre des forces: \\
$$ \vv{F}_{A/B} = K \cdot \dfrac{q_{a}\cdot q_{b}}{r^2} \cdot \vv{U}_{A/B} $$
avec\\
$$K=\dfrac{1}{4\pi\varepsilon_{0}}= 9,0\cdot 10^9(S.I.)$$
$\varepsilon_{0}$: permittivité du vide (réponse d'un milieu donné a un champ électrique appliqué) ($8,85\cdot 10^9$)\\
\danger[5ex] \qquad $\vv{F}$ et $\vv{E}$ n'ont pas forcément le meme sens, cela dépend de la charge $q$\\\\

La relation entre force électrique et champ électrique s'exprime avec $q$ (Coulombs), charge de source:
$$\vv{F_{e}}=q\cdot \vv{E}$$\\
$$\Rightarrow F_{e}=|q|\cdot E$$\\\\
Le champ électrique s'exprime donc de cette maniere:
$$\vv{E}= \dfrac{1}{4\pi\varepsilon_{0}}\cdot\dfrac{q}{r^2}\cdot\vv{U}_{A/B}$$\\
$\vv{E}$ va dans le sens des potentiels décroissants\\
Les lignes de champ sont tangentes aux vecteurs champ électrique tandis que les équipotentielles relient les points ou le champ électrique possede la meme valeur(intensité) \\\\
Dans un condensateur plan, le champ électrique est uniforme (lignes de champ paralleles) et la valeur du champ électrique est
$$E=\dfrac{|U_{ab}|}{d}$$
et, avec $Q$ (charge totale) et $S$ (surface des armatures)
\\
\definecolor{betterPurple}{rgb}{0.6,0,1}
\begin{tikzpicture}
    \draw (0,0) node {};
    \draw (7.85,0) node {$E=\dfrac{Q}{\varepsilon_{0} \cdot S}$};
    \draw[line width=0.4pt] (6.9,-0.05) -- (5.9,-0.45) node[color=blue,left] {$J$};
    \draw[line width=0.4pt] (8.5,0.25) -- (9.5,0.25) node[color=blue,right] {$C$};
    \draw[line width=0.4pt] (8.7,-0.3) -- (9.7,-0.7) node[color=blue,right] {$m^{2}$};
    \draw[line width=0.4pt] (7.7,-0.3) -- (6.7,-1) node[color=betterPurple,left] {$S.I.$};
\end{tikzpicture}

\end{Definition}


\subsection{Le champ magnétique}
