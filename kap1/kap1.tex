\chapter{Folgen}

Eine Funktion , bei der jeder natürlichen Zahl eine reelle Zahl zugeordnet wird, nennt man Folge.\\
Folgen können auch nur für Teilbereiche von $\N$ definiert sein.\\
$(a_{n})_{(n\in\N)}$ bezeichnet die Folge, wobei $\N\rightarrow\R$\\


\section{Verschiedene Darstellungen}


\subsection{Explizite Darstellung}

Wenn ein beliebiges Glied der Folge direkt berechenbar ist, ist ihre Darstellung explizit.

\subsection{Rekursive Darstellung}

Wenn für die Berechnung des n-ten Gliedes einer Folge das (n-1)-te Glied benötigt wird, ist ihre Darstellung rekursiv.


\section{Auffällige Folgen}


\subsection{Arithmetische Folgen}

\subsection{Geometrische Folgen}


\section{Klassifizierung von Folgen}


\subsection{Monotonie}

\subsection{Beschränktheit}

\subsection{Konvergenz}

\subsubsection{Definition}
\subsubsection{Epsilon-n0-Definition}
\subsubsection{Grenzwertsätze}


\section{Vollständige Induktion}
