\chapter{Exponentialfunktionen}

\begin{Definition}
Man bezeichnet als Exponentialfunktion eine Funktion der Form $x\rightarrow a^x$ mit $a\in \R$, $a>0$ und $a\neq1$\\
$x$ ist die Variable und wird \textit{Exponent} oder \textit{Hochzahl} genannt.\\
$a$ nennt man \textit{Basis} oder \textit{Grundzahl}, sie ist für jede Funktion fest forgegeben.\\
Die natürliche Exponentialfunktion wird durch die Funktionsvorschrift $f(x)=e^x$ beschrieben.
\end{Definition}

Hier Graphen für a<1\\
a>1\\
a=e\\


		\section{Wiederholung: Potenzgesetze}
Seien $a$ und $b$ aus $\R$, sowie $n$ und $m$ aus $\N$, dann gilt:
\begin{enumerate}
\begin{minipage}{0.5\textwidth}
\item$a^0=1$ (für $a\neq 0$)
\item$a^1=a$
\item$a^n=\underbrace{a\cdot a\cdot ... \cdot a}_{n Mal}$
\item$a^m\cdot a^n=a^{m+n}$
\item$(a^n)^m=a^{n\cdot m}$
\end {minipage}
\begin{minipage}{0.5\textwidth}
\item$a^{-n}=\dfrac{1}{a^n}$
\item$\dfrac{a^n}{a^m}=a^{n-m}$
\item$\left(\dfrac{a}{b}\right)^n=\dfrac{a^n}{b^n}$
\item $a^{\dfrac{1}{n}}=\sqrt[n]{a}$
\item$a^{\dfrac{m}{n}}=\sqrt[n]{a^m}$
\end {minipage}
\end{enumerate}

		\section{Die Eulersche Zahl ($e$)}

\begin{Definition}
$e = 2,71828182845904523536028747135266249775724709369995...$\\
\\
$e$ ist eine irrationale, transzendente und reelle Zahl, die die Basis des (natürlichen) Logarithmus und der (natürlichen) Exponentialfunktion ist.\\
Die Darstellung, der man am Häufigsten begegnet ist diese:
$e=\lim\limits_{n\to \infty}\left(1+\dfrac{1}{n}\right)^n$
\end{Definition}
Benannt nach dem bekannten Mathematiker Leonhard Euler ist diese Zahl eine der wichtigsten Konstanten der Mathematik.
Sie ist die Basis des natürlichen Logarithmus und der natürlichen Exponentialfunktion. Diese (spezielle) Exponentialfunktion wird aufgrund dieser Beziehung zur Zahl $e$ häufig kurz $e$-Funktion genannt.
\begin{Definition}
  Eine reelle Zahl heißt (oder allgemeiner eine komplexe Zahl) transzendent,
  wenn sie nicht Nullstelle eines Polynoms mit ganzzahligen Koeffizienten ist.
  Andernfalls handelt es sich um eine algebraische Zahl. Jede reelle transzendente Zahl ist überdies irrational.
\end{Definition}

	\subsection{Verschiedene Darstellungen}

$e$ ist darstellbar bzw. ergibt sich durch:
\begin{itemize}
\item $\sum\limits_{k=0}^{\infty}\dfrac{1}{k!}$
\item $\lim\limits_{t\to \infty}\left(1+\dfrac{1}{t}\right)^t;t\in\R$
\item $\lim\limits_{n\to \infty}\left(1+\dfrac{1}{n}\right)^n;n\in\N$
\end{itemize}

	\subsection{Herleitung zur Zahl $e$}

Anhand mancher Überlegungen, denen wir jetzt nachgehen werden, lassen sich einige Eigenschaften der Eulerschen Zahl schließen.\\
\\ Wir definieren eine Folge $(e_{n})n\in\N$ durch $e_{n}=\left(1+\dfrac{1}{n}\right)^n$ und versuchen ihre Konvergenz zu beweisen.
\begin{itemize}
\item$\dfrac{e_{n+1}}{e_{n}}=\dfrac{{\left(1+\dfrac{1}{n+1}\right)}^{n+1}}{{\left(1+\dfrac{1}{n+1}\right)}^{n}}=\left(\dfrac{n(n+2)}{(n+1)^2}\right)^n\cdot\left(1+\dfrac{1}{n}\right)=\left(1-\dfrac{1}{(n+1)^2}\right)^{n+1}\cdot\dfrac{n+1}{n}$
\item Die Umformung ermöglicht uns auf den Term ${\left(1-\dfrac{1}{(n+1)^2}}\right)^{n+1}$, die Ungleichung von Bernoulli anzuwenden. Diese besagt Folgendes: $(1+x)^n>1+nx$ für $n\geq2$ und $x>-1$\\
$\left(1-\dfrac{1}{(n+1)^2}\right)^{n+1}>1+(n+1)\cdot\left(-\dfrac{1}{(n+1)^2}\right)$
\item Dies kann man an den ersten Ausdruck anwenden:\\
$\Rightarrow\left(1-\dfrac{1}{(n+1)^2}\right)^{n+1}\cdot\dfrac{n+1}{n}>1+(n+1)\cdot\left(-\dfrac{1}{(n+1)^2}\right)\cdot\dfrac{n+1}{n}=1-\dfrac{1}{n}$

\end{itemize}

		\section{Eigenschaften}
		\section{Ableitungsregeln}

\begin{enumerate}
\item 
\end{enumerate}
	\subsection{Aktivität}

Quelle: Déclic 1ère

