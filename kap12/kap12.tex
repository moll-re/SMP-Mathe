\chapter{Exponetialfunktionen}
\section{Die Zahl $e$}
\begin{Definition}
$e = 2,71828182845904523536028747135266249775724709369995...$\\
Ist eine \textbf{irrationale} und \textbf{transzendete} reelle Zahl, entdeckt von Leonard Euler.Sie ist die Basis des natürlichen Logarithmus und der (natürlichen) Exponentialfunktion.
Diese (spezielle) Exponentialfunktion wird aufgrund dieser Beziehung zur Zahl $e$ häufig kurz $e$Funktion genannt.\\
Die Eulersche Zahl spielt in der gesamten Analysis und allen damit verbundenen Teilgebieten der Mathematik, besonders in der Differential- und Integralrechnung, eine zentrale Rolle.
Sie gehört zu den wichtigsten Konstanten der Mathematik.
\end{Definition}
\begin{Definition}
  Eine reelle Zahl heißt (oder allgemeiner eine komplexe Zahl) transzendent,
  wenn sie nicht Nullstelle eines Polynoms mit ganzzahligen Koeffizienten ist.
  Andernfalls handelt es sich um eine algebraische Zahl. Jede reelle transzendente Zahl ist überdies irrational.
\end{Definition}
\subsection{Darstellung}
$e$ ist darstellbar bzw. ergibt sich durch:
\begin{itemize}
  \item \sum\limits_{k=0}^{\infty}\dfrac{1}{k!}
  \item
\end{itemize}
