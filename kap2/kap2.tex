\chapter{Reihen}

\begin{Definition}
Eine Reihe ist eine Folge, deren Glieder die Partialsummen einer anderen Folge ist. Das bedeutet, dass das $n-te$ Glied der Reihe, die Summe der ersten $n$ Glieder einer anderen Folge ist. \\
Man hat also:

\begin{enumerate} 
\item Mit Startglied $a_{0}$: $s_{n}=\sum\limits_{i=0}^{n-1}a_{i}$
\item Mit Startglied $a_{1}$: $s_{n}=\sum\limits_{i=1}^{n}a_{i}$
\item Mit Startglied $a_{x}$: $s_{n}=\sum\limits_{i=x}^{n+x-1}a_{i}$
\end{enumerate}

\end{Definition}

\begin{Bemerkung}
In manchen Fällen steht $s_{n}$ für die Partialsumme einer anderen Folge bis zum $n-ten$ Glied.
Dann gilt für ein beliebiges Startglied $a_{x}$ der Folge: $s_{n}=\sum\limits_{i=x}^{n}a_{i}$
\end{Bemerkung}

		\section{Artithmetische Reihen}

